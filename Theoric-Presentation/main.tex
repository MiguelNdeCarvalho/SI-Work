\documentclass{beamer}
\usepackage[utf8]{inputenc}
\usepackage[portuguese]{babel}
\usepackage{pgf}
\usepackage{ragged2e}

\setbeamertemplate{caption}[numbered]

\title[Segurança Informática]
{\textbf{Distribuições Linux para Segurança}}

\subtitle{Segurança Informática}

\institute[UE]

\author[Miguel Carvalho, João Pereira]
{\textbf{Professor:} Pedro Patinho\\
\textbf{Realizado por:} Miguel Carvalho(43108) e João Pereira(42864)}

\date[2020-2021]
{\today}

\usebackgroundtemplate{\includegraphics[height=\textwidth]{background.jpg}}

\logo{\pgfputat{\pgfxy(-0.23,7.5)}{\pgfbox[right,base]{\includegraphics[width=.33\linewidth]{di.pdf}}}}
\newcommand{\nologo}{\setbeamertemplate{logo}{}}

\usetheme{Madrid}

\begin{document}
\maketitle
\nologo
%-----------------------------------------------------------------------------------
\begin{frame}{O que são?}

\justifying
As \textbf{Distribuições Linux para Segurança} são distribuições Linux convencionais que apresentam um grande foco em segurança. O que principalmente as distingue de qualquer outra distribuição Linux é o facto de conterem várias ferramentas pré-instaladas e pré-configuradas que são necessárias para realizar testes de segurança e outras tarefas semelhantes.  

\end{frame}
%-----------------------------------------------------------------------------------
\begin{frame}{Ferramentas}

As ferramentas encontram-se divididas em grupos, que são:
\begin{itemize}
    \item Information Gathering
    \item Vulnerability Analysis
    \item Wireless Attacks
    \item Web Applications
    \item Exploitation Tools
    \item Stress Testing
    \item Forensics Tools
    \item Sniffing and Spoofing
    \item Password Attacks
    \item Maintaining Access
    \item Reverse Engineering
    \item Reporting Tools
    \item Hardware Hacking
\end{itemize}

\end{frame}
%-----------------------------------------------------------------------------------
\begin{frame}{Ferramentas: Information Gathering}

Neste grupo estão todas as ferramentas que permitem \textbf{obter informações}.

\hfill

Algumas das ferramentas pertencentes a este grupo:
    \begin{itemize}
        \justifying
        \item \textbf{Nmap} - é um \textit{port scan} utilizado para descobrir serviços ou servidores numa rede de computadores;
    \item \textbf{Sublist3r} - é uma ferramenta que permite listar os subdomínios de websites utilizando principalmente o OSINT;
        \item \textbf{WHOIS} - é um protocolo TCP/IP utilizado para consultar informações de contato e DNS sobre entidades na Internet.
        \item \textbf{DIG} - é utilizado para obter informação sobre o DNS
    \end{itemize}
\end{frame}
%-----------------------------------------------------------------------------------
\begin{frame}{Ferramentas: Vulnerability Analysis}

Neste grupo estão todas as ferramentas que permitem \textbf{analisar informações}.

\hfill

Algumas das ferramentas pertencentes a este grupo:
    \begin{itemize}
        \justifying
        \item \textbf{BED} - projetada para verificar as \textit{daemons} em busca de potenciais buffer overflows;
        \item \textbf{Osscanner} - é uma framework de acesso Oracle desenvolvida em Java;
        \item \textbf{sqlmap} - teste de penetração que automatiza o processo de deteção e exploração de falhas através de SQL Injection e a obtenção do controlo de servidores de base de dados;
        \item \textbf{SIPArmyKnife} - pesquisa de cross site scripting, SQL Injection, log injection, format string, buffer overlows, entre outras.
    \end{itemize}
\end{frame}
%-----------------------------------------------------------------------------------
\begin{frame}{Ferramentas: Wireless Attacks}

Neste grupo estão todas as ferramentas que permitem \textbf{atacar dispositivos Wireless}.

\hfill

Algumas das ferramentas pertencentes a este grupo:
    \begin{itemize}
        \item \textbf{Aircrack-ng} - cracking de chaves 802.11 WEP e WPA-PSK que consegue recuperar chaves a partir de informação dos pacotes capturados;
        \item \textbf{Kismet} - usada para fazer sniff de pacotes TCP, UDP, DHCP e ARP;
        \item \textbf{coWPAtty} - permite realizar ataques em redes que utilizam o WPA com chaves pré-partilhadas;
        \item \textbf{Reaver} - usa técninas de força bruta contra WiFis que usam PINs para obter as palavra-passes WPA/WPA2.
    \end{itemize}
\end{frame}
%-----------------------------------------------------------------------------------
\begin{frame}{Ferramentas: Web Applications}

Neste grupo estão todas as ferramentas que permitem \textbf{explorar as aplicações web}.

\hfill

Algumas das ferramentas pertencentes a este grupo:
    \begin{itemize}
        \item \textbf{Zed Attack Proxy (ZAP)} - utilizada para encontrar vulnerabilidades em aplicações web;
        \item \textbf{Wfuzz} - utilizada para realizar ataques de força bruta em aplicações web, pode ser usada para encontrar recursos não ligados (diretórios, scripts, entre outros);
        \item \textbf{W3af} - tem como principal objetivo identificar e explorar vulnerabilidades em aplicações web;
        \item \textbf{Grabber} - procura por vulenrabilidades em aplicações web.
    \end{itemize}
\end{frame}
%-----------------------------------------------------------------------------------
\begin{frame}{Ferramentas: Exploitation Tools}

Neste grupo estão todas as ferramentas que permitem \textbf{explorar falhas}.

\hfill

Algumas das ferramentas pertencentes a este grupo:
    \begin{itemize}
        \item \textbf{Metasploit Framework} - é uma plataforma de testes de penetração que permite encontrar, explorar e validar falhas;
        \item \textbf{cisco-auditing-tool} - capaz de procurar por vulnerabilidades comuns em routers da Cisco;
        \item \textbf{exploitdb} - permite pesquisar falhas a partir da \textbf{The Exploit Database};
        \item \textbf{Linux Exploit Suggester} - simples script que se mantem atualizado em relação a falhas e sugere possivéis usos de falhas para ganhar acesso root do sistema.
    \end{itemize}
\end{frame}
%-----------------------------------------------------------------------------------
\begin{frame}{Ferramentas: Stress Testing}

Neste grupo estão todas as ferramentas que permitem realizar \textbf{testes de stress}.

\hfill

Algumas das ferramentas pertencentes a este grupo:
    \begin{itemize}
        \item \textbf{DHCPig} - permite attacar um servidor DHCP ao consumir todos os IPs disponíveis;
        \item \textbf{FunkLoad} - permite testar aplicações web causando um load no servidor;
        \item \textbf{inviteflood} - envia mensagens fazendo flood através de UDP/IP;
        \item \textbf{rtpflood} - faz um ataque de RTP a qualquer dispositivo que processa RTP.
    \end{itemize}
\end{frame}
%-----------------------------------------------------------------------------------
\begin{frame}{Ferramentas: Forensics Tools}

Neste grupo estão todas as ferramentas que permitem realizar \textbf{testes de forence}.

\hfill

Algumas das ferramentas pertencentes a este grupo:
    \begin{itemize}
        \item \textbf{bulk-extractor} - extrai informações como mail, credit card, URLs e outro tipo de informação a partir de ficheiros de evidencias;
        \item \textbf{Cuckoo} - analisador de malwares;
        \item \textbf{Dumpzilla} - tem como objetivo extrair toda a informação interessante dos browsers Firefox, Iceweasel e do Seamonkey para ser analisado;
        \item \textbf{iPhone-Backup-Analyzer} - permite aceder aos ficheiros de configurações, aos ficheiros do utilizador e analisar a base de dados de um iPhone ou outro dispositivo IOS.
    \end{itemize}
\end{frame}
%-----------------------------------------------------------------------------------
\begin{frame}{Ferramentas: Sniffing and Spoofing}

Neste grupo estão todas as ferramentas que permitem realizar \textbf{Sniffing} e \textbf{Spoofing}.

\hfill

Algumas das ferramentas pertencentes a este grupo:
    \begin{itemize}
        \item \textbf{Bettercap} - é um canivete suíço para ataques na rede e monitorização;
        \item \textbf{Burp Suite} - permite realizar testes de segurança em aplicações web;
        \item \textbf{iSMTP} - usado para testar servidores SMTP;
        \item \textbf{rtpbreak} - permite detetar, reconstruir e analisar uma sessão RTP.
    \end{itemize}
\end{frame}
%-----------------------------------------------------------------------------------
\begin{frame}{Ferramentas: Password Attacks}

Neste grupo estão todas as ferramentas que permitem realizar \textbf{ataques em passwords}.

\hfill

Algumas das ferramentas pertencentes a este grupo:
    \begin{itemize}
        \item \textbf{BruteSpray} - utiliza força bruta ao tentar utilizar as credencias predefinidas pelos serviços;
        \item \textbf{CmosPwd} - permite desencriptar a palavra-passe guardada na CMOS usada para aceder à BIOS;
        \item \textbf{crowbar} - usa força bruta e pode ser usada para testes de penetração;
        \item \textbf{RainbowCrack} - usado para fazer crack de hashes através de rainbow tables.
    \end{itemize}
\end{frame}
%-----------------------------------------------------------------------------------
\begin{frame}{Ferramentas: Maintaining Access}

Neste grupo estão todas as ferramentas que permitem \textbf{manter o acesso}.

\hfill

Algumas das ferramentas pertencentes a este grupo:
    \begin{itemize}
        \item \textbf{CryptCat} - permite ler e escrever dados em conexões encriptadas utilizando TCP ou UDP;
        \item \textbf{Cymothoa} - permite injetar código num processo existente;
        \item \textbf{Dns2tcp} - tem como principal objetivo retransmitir conexões TCP através de tráfego DNS;
        \item \textbf{pwnat} - permite que qualquer número de utilizadores atrás de uma rede NAT comuniquem com um servidor atrás de outra rede NAT.
    \end{itemize}
\end{frame}
%-----------------------------------------------------------------------------------
\begin{frame}{Ferramentas: Reverse Engineering}

Neste grupo estão todas as ferramentas que permitem fazer \textbf{engenharia reversa}.

\hfill

Algumas das ferramentas pertencentes a este grupo:
    \begin{itemize}
        \item \textbf{apktool} - permite fazer Reverse Engineering numa aplicação Android;
        \item \textbf{dex2jar} - permite converter Dalvik Executable (.dex) para (.jar);
        \item \textbf{edb-debugger} - permite fazer debug de APIs;
        \item \textbf{YARA} - permite criar descrições de familias de malwares baseados em texto ou padrões binários.
    \end{itemize}
\end{frame}
%-----------------------------------------------------------------------------------
\begin{frame}{Ferramentas: Reporting Tools}

Neste grupo estão todas as ferramentas que permitem criar \textbf{relatórios}.

\hfill

Algumas das ferramentas pertencentes a este grupo:
    \begin{itemize}
        \item \textbf{CaseFile} - permite contruir gráficos com dados guardados a partir de investigações;
        \item \textbf{dos2unix} - permite converter ficheiros de dos para unix;
        \item \textbf{pipal} - mostra o estado e a informação para ajudar a analisar passwords;
        \item \textbf{python-rdpy} - implemetanção do Microsoft RDP em Python.
    \end{itemize}
\end{frame}
%-----------------------------------------------------------------------------------
\begin{frame}{Ferramentas: Hardware Hacking}

Neste grupo estão todas as ferramentas que permitem utilizar \textbf{bibliotecas}.

\hfill

Algumas das ferramentas pertencentes a este grupo:
    \begin{itemize}
        \item \textbf{android-sdk} - contêm as bibliotecas para desenvolver para Android;
        \item \textbf{Arduino} - contêm as bibliotecas para desenvolver para placas Arduino;
        \item \textbf{Sakis3G} - permite estabelecer conexões 3g;
        \item \textbf{smali} - assembler para dex usado pelo dalvik.
    \end{itemize}
\end{frame}
%-----------------------------------------------------------------------------------
\begin{frame}{Distribuições Linux para Pentesting}

\begin{itemize}
    \item ArchStrike
    \item BackBox
    \item BlackArch
    \item KaliLinux
    \item ParrotSecurityOS
\end{itemize}

\end{frame}
%-----------------------------------------------------------------------------------
\begin{frame}{ArchStrike}

\begin{figure}[h]
    \includegraphics[width=0.2\textwidth]{distros/archstrike.png}
    \centering
\end{figure}

\begin{itemize}
    \item Antigamente denominado por \textbf{ArchAssault}
    \item Baseado em: \textbf{Arch Linux}
    \item Ambiente gráfico: \textbf{Openbox}
    \item Estado: \textbf{Em ativo desenvolvimento}
\end{itemize}

\end{frame}
%-----------------------------------------------------------------------------------
\begin{frame}{BackBox Linux}

\begin{figure}[h]
    \includegraphics[width=0.2\textwidth]{distros/backbox.png}
    \centering
\end{figure}

\begin{itemize}
    \item Baseado em: \textbf{Ubuntu, Debian}
    \item Ambiente gráfico: \textbf{Xfce}
    \item Estado: \textbf{Em ativo desenvolvimento}
\end{itemize}

\end{frame}
%-----------------------------------------------------------------------------------
\begin{frame}{BlackArch Linux}

\begin{figure}[h]
    \includegraphics[width=0.2\textwidth]{distros/blackarch.png}
    \centering
\end{figure}

\begin{itemize}
    \item Baseado em: \textbf{Arch}
    \item Ambiente gráfico: \textbf{Awesome, Blackbox, Fluxbox, spectrwn}
    \item Estado: \textbf{Em ativo desenvolvimento}
\end{itemize}

\end{frame}
%-----------------------------------------------------------------------------------
\begin{frame}{Kali Linux}

\begin{figure}[h]
    \includegraphics[width=0.2\textwidth]{distros/kali.png}
    \centering
\end{figure}

\begin{itemize}
    \item Baseado em: \textbf{Debian(Testing)}
    \item Ambiente gráfico: \textbf{Enlightenment, GNOME, KDE Plasma, LXDE, MATE, Xfce(Default)}
    \item Estado: \textbf{Em ativo desenvolvimento}
\end{itemize}
    
\end{frame}
%-----------------------------------------------------------------------------------
\begin{frame}{ParrotSecurityOS}

\begin{figure}[h]
    \includegraphics[width=0.2\textwidth]{distros/parrot.png}
    \centering
\end{figure}
    
\begin{itemize}
    \item Baseado em: \textbf{Debian(Testing)}
    \item Ambiente gráfico: \textbf{KDE Plasma e MATE}
    \item Estado: \textbf{Em ativo desenvolvimento}
\end{itemize}
    
\end{frame}
%-----------------------------------------------------------------------------------
\end{document}